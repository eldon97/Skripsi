\chapter{Kode Perangkat Lunak \textit{Randomization}}
\label{lamp:A}

\newcommand{\package}[2]{
	\section{\textit{Package #1}}
	\label{sec:p-#1}

	Dalam \textit{package #1} terdapat #2 buah kelas. Kode program untuk masing-masing kelas dalam \textit{package} ini adalah sebagai berikut.
}

\package{Perturbation}{3}

\subsection{Kelas \textit{Perturbation}}
\label{subsec:c-perturbation}
Kode program untuk kelas \textit{Perturbation} adalah sebagai berikut.
\lstinputlisting[language=Python, caption=perturbation.py]{./Lampiran/src/randomization/perturbation/perturbation.py}

\subsection{Kelas \textit{RandomRotationPerturbation}}
\label{subsec:c-rotation}
Kode program untuk kelas \textit{RandomRotationPerturbation} adalah sebagai berikut.
\lstinputlisting[language=Python, caption=random\_rotation\_perturbation.py]{./Lampiran/src/randomization/perturbation/random_rotation_perturbation.py}

\subsection{Kelas \textit{RandomProjectionPerturbation}}
\label{subsec:c-projection}
Kode program untuk kelas \textit{RandomProjectionPerturbation} adalah sebagai berikut.
\lstinputlisting[language=Python, caption=random\_projection\_perturbation.py]{./Lampiran/src/randomization/perturbation/random_projection_perturbation.py}

\package{Matrix}{4}

\subsection{Kelas \textit{Matrix}}
\label{subsec:c-matrix}
Kode program untuk kelas \textit{Matrix} adalah sebagai berikut.
\lstinputlisting[language=Python, caption=matrix.py]{./Lampiran/src/randomization/matrix/matrix.py}

\subsection{Kelas \textit{RandomRotationMatrix}}
\label{subsec:c-rotation-matrix}
Kode program untuk kelas \textit{RandomRotationMatrix} adalah sebagai berikut.
\lstinputlisting[language=Python, caption=random\_rotation\_matrix.py]{./Lampiran/src/randomization/matrix/random_rotation_matrix.py}

\subsection{Kelas \textit{RandomTranslationMatrix}}
\label{subsec:c-translation-matrix}
Kode program untuk kelas \textit{RandomTranslationMatrix} adalah sebagai berikut.
\lstinputlisting[language=Python, caption=random\_translation\_matrix.py]{./Lampiran/src/randomization/matrix/random_translation_matrix.py}

\subsection{Kelas \textit{RandomProjectionMatrix}}
\label{subsec:c-projection-matrix}
Kode program untuk kelas \textit{RandomProjectionMatrix} adalah sebagai berikut.
\lstinputlisting[language=Python, caption=random\_projection\_matrix.py]{./Lampiran/src/randomization/matrix/random_projection_matrix.py}

\package{Preprocessor}{3}

\subsection{Kelas \textit{CSVPreprocessor}}
\label{subsec:c-csv-pre}
Kode program untuk kelas \textit{CSVPreprocessor} adalah sebagai berikut.
\lstinputlisting[language=Python, caption=csv\_preprocessor.py]{./Lampiran/src/randomization/preprocessor/csv_preprocessor.py}

\subsection{Kelas \textit{ProjectionMatrixPreprocessor}}
\label{subsec:c-projection-pre}
Kode program untuk kelas \textit{ProjectionMatrixPreprocessor} adalah sebagai berikut.
\lstinputlisting[language=Python, caption=projection\_matrix\_preprocessor.py]{./Lampiran/src/randomization/preprocessor/projection_matrix_preprocessor.py}

\subsection{Kelas \textit{RotationMatrixPreprocessor}}
\label{subsec:c-rotation-pre}
Kode program untuk kelas \textit{ProjectionMatrixPreprocessor} adalah sebagai berikut.
\lstinputlisting[language=Python, caption=rotation\_matrix\_preprocessor.py]{./Lampiran/src/randomization/preprocessor/rotation_matrix_preprocessor.py}

\section{\textit{Package View}}
\label{sec:p-view}

Dalam \textit{package View} terdapat beberapa buah kelas yang berfungsi untuk menangani antarmuka perangkat lunak yang dibuat dengan \textit{framework} Kivy. Kode program untuk masing-masing dokumen dalam \textit{package} ini adalah sebagai berikut.

\subsection{Dokumen \textit{main\_menu.py}}
\label{subsec:c-main}
Kode program untuk dokumen \textit{main\_menu.py} adalah sebagai berikut.
\lstinputlisting[language=Python, caption=main\_menu.py]{./Lampiran/src/randomization/view/main_menu.py}

\subsection{Dokumen \textit{randomization\_app.py}}
\label{subsec:c-randomization-app}
Kode program untuk dokumen \textit{randomization\_app.py} adalah sebagai berikut.
\lstinputlisting[language=Python, caption=randomization\_app.py]{./Lampiran/src/randomization/view/randomization_app.py}

\subsection{Dokumen \textit{randomization.kv}}
\label{subsec:c-randomization}
Kode program untuk dokumen \textit{randomization.kv} adalah sebagai berikut.
\lstinputlisting[language={}, caption=randomization.kv]{./Lampiran/src/randomization/view/randomization.kv}