\chapter{Kode Perangkat Lunak Pengujian}
\label{lamp:B}

\section{Pengujian Penambangan Data Klasifikasi \textit{k-nearest neighbors}}
\label{sec:pengujian-knn}

Kode program untuk pengujian penambangan data klasifikasi dibagi menjadi 2 bagian yaitu sebagai berikut.

\subsection{\textit{Random Rotation Perturbation}}
\label{subsec:knn-rotation-perturbation}
Kode program untuk pengujian penambangan data klasifikasi terhadap teknik \textit{Random Rotation Perturbation} dengan dataset \textit{diabetes} adalah sebagai berikut.
\lstinputlisting[language=Python, caption=knn\_diabetes.py]{./Lampiran/src/pengujian/knn_diabetes.py}

\subsection{\textit{Random Projection Perturbation}}
\label{subsec:knn-projection-perturbation}
Kode program untuk pengujian penambangan data klasifikasi terhadap teknik \textit{Random Rotation Perturbation} dengan dataset \textit{mobile\_sensor} adalah sebagai berikut.
\lstinputlisting[language=Python, caption=knn\_mobile\_sensor.py]{./Lampiran/src/pengujian/knn_mobile_sensor.py}

\section{Pengujian Penambangan Data \textit{Clustering k-means}}
\label{sec:pengujian-kmeans}

Kode program untuk pengujian penambangan data \textit{clustering} dibagi menjadi 2 bagian yaitu sebagai berikut.

\subsection{\textit{Random Rotation Perturbation}}
\label{subsec:kmeans-rotation-perturbation}
Kode program untuk pengujian penambangan data \textit{clustering} terhadap teknik \textit{Random Rotation Perturbation} dengan dataset \textit{mall\_customers} adalah sebagai berikut.
\lstinputlisting[language=Python, caption=kmeans\_mall.py]{./Lampiran/src/pengujian/kmeans_mall.py}

\subsection{\textit{Random Projection Perturbation}}
\label{subsec:kmeans-projection-perturbation}
Kode program untuk pengujian penambangan data \textit{clustering} terhadap teknik \textit{Random Projection Perturbation} dengan dataset \textit{mobile\_sensor} adalah sebagai berikut.
\lstinputlisting[language=Python, caption=kmeans\_mobile\_sensor.py]{./Lampiran/src/pengujian/kmeans_mobile_sensor.py}

\section{Pengujian Lainnya}
\label{sec:pengujian-lainnya}

Kode program untuk pengujian lainnya seperti pengujian jarak Euclidean dan \textit{Adjusted Rand Index} adalah sebagai berikut.

\lstinputlisting[language=Python, caption=pengujian.py]{./Lampiran/src/pengujian/pengujian.py}