%_____________________________________________________________________________
%=============================================================================
% data.tex v10 (22-01-2017) dibuat oleh Lionov - T. Informatika FTIS UNPAR
%
% Perubahan pada versi 10 (22-01-2017)
%	- Penambahan overfullrule untuk memeriksa warning
%  	- perubahan mode buku menjadi 4: bimbingan, sidang(1), sidang akhir dan 
%     buku final
%	- perbaikan perintah pada beberapa bagian
%  	- perubahan pengisian tulisan "daftar isi" yang error
%  	- penghilangan lipsum dari file ini
%_____________________________________________________________________________
%=============================================================================

%=============================================================================
% 								PETUNJUK
%=============================================================================
% Ini adalah file data (data.tex)
% Masukkan ke dalam file ini, data-data yang diperlukan oleh template ini
% Cara memasukkan data dijelaskan di setiap bagian
% Data yang WAJIB dan HARUS diisi dengan baik dan benar adalah SELURUHNYA !!
% Hilangkan tanda << dan >> jika anda menemukannya
%=============================================================================

%_____________________________________________________________________________
%=============================================================================
% 								BAGIAN 0
%=============================================================================
% Entri untuk memperbaiki posisi "DAFTAR ISI" jika tidak berada di bagian 
% tengah halaman. Sayangnya setiap sistem menghasilkan posisi yang berbeda.
% Isilah dengan 0 atau 1 (e.g. \daftarIsiError{1}). 
% Pemilihan 0 atau 1 silahkan disesuaikan dengan hasil PDF yang dihasilkan.
%=============================================================================
\daftarIsiError{0}   
%\daftarIsiError{1}   
%=============================================================================

%_____________________________________________________________________________
%=============================================================================
% 								BAGIAN I
%=============================================================================
% Tambahkan package2 lain yang anda butuhkan di sini
%=============================================================================
\usepackage{booktabs} 
\usepackage{longtable}
\usepackage{amssymb}
\usepackage{todo}
\usepackage{verbatim} 		%multiline comment
\usepackage{pgfplots}
\usepackage{amsthm}
\usepackage{amsmath}
\usepackage[boxruled]{algorithm2e}
%\overfullrule=3mm % memperlihatkan overfull 
%=============================================================================

%_____________________________________________________________________________
%=============================================================================
% 								BAGIAN II
%=============================================================================
% Mode dokumen: menetukan halaman depan dari dokumen, apakah harus mengandung 
% prakata/pernyataan/abstrak dll (termasuk daftar gambar/tabel/isi) ?
% - final 		: hanya untuk buku skripsi, dicetak lengkap: judul ina/eng, 
%   			  pengesahan, pernyataan, abstrak ina/eng, untuk, kata 
%				  pengantar, daftar isi (daftar tabel dan gambar tetap 
%				  opsional dan dapat diatur), seluruh bab dan lampiran.
%				  Otomatis tidak ada nomor baris dan singlespacing
% - sidangakhir	: buku sidang akhir = buku final - (pengesahan + pernyataan +
%   			  untuk + kata pengantar)
%				  Otomatis ada nomor baris dan onehalfspacing 
% - sidang 		: untuk sidang 1, buku sidang = buku sidang akhir - (judul 
%				  eng + abstrak ina/eng)
%				  Otomatis ada nomor baris dan onehalfspacing
% - bimbingan	: untuk keperluan bimbingan, hanya terdapat bab dan lampiran
%   			  saja, bab dan lampiran yang hendak dicetak dapat ditentukan 
%				  sendiri (nomor baris dan spacing dapat diatur sendiri)
% Mode default adalah 'template' yang menghasilkan isian berwarna merah, 
% aktifkan salah satu mode di bawah ini :
%=============================================================================
%\mode{bimbingan} 		% untuk keperluan bimbingan
%\mode{sidang} 			% untuk sidang 1
%\mode{sidangakhir} 	% untuk sidang 2 / sidang pada Skripsi 2(IF)
\mode{final} 			% untuk mencetak buku skripsi 
%=============================================================================

%_____________________________________________________________________________
%=============================================================================
% 								BAGIAN III
%=============================================================================
% Line numbering: penomoran setiap baris, nomor baris otomatis di-reset ke 1
% setiap berganti halaman.
% Sudah dikonfigurasi otomatis untuk mode final (tidak ada), mode sidang (ada)
% dan mode sidangakhir (ada).
% Untuk mode bimbingan, defaultnya ada (\linenumber{yes}), jika ingin 
% dihilangkan, isi dengan "no" (i.e.: \linenumber{no})
% Catatan:
% - jika nomor baris tidak kembali ke 1 di halaman berikutnya, compile kembali
%   dokumen latex anda
% - bagian ini hanya bisa diatur di mode bimbingan
%=============================================================================
\linenumber{no} 
%\linenumber{yes}
%=============================================================================

%_____________________________________________________________________________
%=============================================================================
% 								BAGIAN IV
%=============================================================================
% Linespacing: jarak antara baris 
% - single	: otomatis jika ingin mencetak buku skripsi, opsi yang 
%			     disediakan untuk bimbingan, jika pembimbing tidak keberatan 
%			     (untuk menghemat kertas)
% - onehalf	: otomatis jika ingin mencetak dokumen untuk sidang
% - double 	: jarak yang lebih lebar lagi, jika pembimbing berniat memberi 
%             catatan yg banyak di antara baris (dianjurkan untuk bimbingan)
% Catatan: bagian ini hanya bisa diatur di mode bimbingan
%=============================================================================
\linespacing{single}
%\linespacing{onehalf}
%\linespacing{double}
%=============================================================================

%_____________________________________________________________________________
%=============================================================================
% 								BAGIAN V
%=============================================================================
% Tidak semua skripsi memuat gambar dan/atau tabel. Untuk skripsi yang tidak 
% memiliki gambar dan/atau tabel, maka tidak diperlukan Daftar Gambar dan/atau 
% Daftar Tabel. Sayangnya hal tsb sulit dilakukan secara manual karena 
% membutuhkan kedisiplinan pengguna template.  
% Jika tidak ingin menampilkan Daftar Gambar dan/atau Daftar Tabel, karena 
% tidak ada gambar atau tabel atau karena dokumen dicetak hanya untuk 
% bimbingan, isi dengan "no" (e.g. \gambar{no})
%=============================================================================
\gambar{yes}
%\gambar{no}
\tabel{yes}
%\tabel{no}  
%=============================================================================

%_____________________________________________________________________________
%=============================================================================
% 								BAGIAN VI
%=============================================================================
% Pada mode final, sidang da sidangkahir, seluruh bab yang ada di folder "Bab"
% dengan nama file bab1.tex, bab2.tex s.d. bab9.tex akan dicetak terurut, 
% apapun isi dari perintah \bab.
% Pada mode bimbingan, jika ingin:
% - mencetak seluruh bab, isi dengan 'all' (i.e. \bab{all})
% - mencetak beberapa bab saja, isi dengan angka, pisahkan dengan ',' 
%   dan bab akan dicetak terurut sesuai urutan penulisan (e.g. \bab{1,3,2}). 
% Catatan: Jika ingin menambahkan bab ke-3 s.d. ke-9, tambahkan file 
% bab3.tex, bab4.tex, dst di folder "Bab". Untuk bab ke-10 dan 
% seterusnya, harus dilakukan secara manual dengan mengubah file skripsi.tex
% Catatan: bagian ini hanya bisa diatur di mode bimbingan
%=============================================================================
\bab{all}
%=============================================================================

%_____________________________________________________________________________
%=============================================================================
% 								BAGIAN VII
%=============================================================================
% Pada mode final, sidang dan sidangkhir, seluruh lampiran yang ada di folder 
% "Lampiran" dengan nama file lampA.tex, lampB.tex s.d. lampJ.tex akan dicetak 
% terurut, apapun isi dari perintah \lampiran.
% Pada mode bimbingan, jika ingin:
% - mencetak seluruh lampiran, isi dengan 'all' (i.e. \lampiran{all})
% - mencetak beberapa lampiran saja, isi dengan huruf, pisahkan dengan ',' 
%   dan lampiran akan dicetak terurut sesuai urutan (e.g. \lampiran{A,E,C}). 
% - tidak mencetak lampiran apapun, isi dengan "none" (i.e. \lampiran{none})
% Catatan: Jika ingin menambahkan lampiran ke-C s.d. ke-I, tambahkan file 
% lampC.tex, lampD.tex, dst di folder Lampiran. Untuk lampiran ke-J dan 
% seterusnya, harus dilakukan secara manual dengan mengubah file skripsi.tex
% Catatan: bagian ini hanya bisa diatur di mode bimbingan
%=============================================================================
\lampiran{all}
%=============================================================================

%_____________________________________________________________________________
%=============================================================================
% 								BAGIAN VIII
%=============================================================================
% Data diri dan skripsi/tugas akhir
% - namanpm		: Nama dan NPM anda, penggunaan huruf besar untuk nama harus 
%				  benar dan gunakan 10 digit npm UNPAR, PASTIKAN BAHWA 
%				  BENAR !!! (e.g. \namanpm{Jane Doe}{1992710001}
% - judul 		: Dalam bahasa Indonesia, perhatikan penggunaan huruf besar, 
%				  judul tidak menggunakan huruf besar seluruhnya !!! 
% - tanggal 	: isi dengan {tangga}{bulan}{tahun} dalam angka numerik, 
%				  jangan menuliskan kata (e.g. AGUSTUS) dalam isian bulan.
%			  	  Tanggal ini adalah tanggal dimana anda akan melaksanakan 
%				  sidang ujian akhir skripsi/tugas akhir
% - pembimbing	: pembimbing anda, lihat daftar dosen di file dosen.tex
%				  jika pembimbing hanya 1, kosongkan parameter kedua 
%				  (e.g. \pembimbing{\JND}{} ), \JND adalah kode dosen
% - penguji 	: para penguji anda, lihat daftar dosen di file dosen.tex
%				  (e.g. \penguji{\JHD}{\JCD} )
% !!Lihat singkatan pembimbing dan penguji anda di file dosen.tex!!
% Petunjuk: hilangkan tanda << & >>, dan isi sesuai dengan data anda
%=============================================================================
\namanpm{Chris Eldon}{2016730073}
\tanggal{12}{6}{2020}
\pembimbing{\MTA}{}    
\penguji{\RDL}{\LCA} 
%=============================================================================

%_____________________________________________________________________________
%=============================================================================
% 								BAGIAN IX
%=============================================================================
% Judul dan title : judul bhs indonesia dan inggris
% - judulINA: judul dalam bahasa indonesia
% - judulENG: title in english
% Petunjuk: 
% - hilangkan tanda << & >>, dan isi sesuai dengan data anda
% - langsung mulai setelah '{' awal, jangan mulai menulis di baris bawahnya
% - gunakan \texorpdfstring{\\}{} untuk pindah ke baris baru
% - judul TIDAK ditulis dengan menggunakan huruf besar seluruhnya !!
%=============================================================================
\judulINA{Analisis Teknik \textit{Random Rotation Perturbation} dan \textit{Random Projection Perturbation} dalam Mengacak Data untuk Penambangan Data}
\judulENG{Analysis of Random Rotation Perturbation and Random Projection Perturbation Techniques in Randomizing Data for Data Mining}
%_____________________________________________________________________________
%=============================================================================
% 								BAGIAN X
%=============================================================================
% Abstrak dan abstract : abstrak bhs indonesia dan inggris
% - abstrakINA: abstrak bahasa indonesia
% - abstrakENG: abstract in english 
% Petunjuk: 
% - hilangkan tanda << & >>, dan isi sesuai dengan data anda
% - langsung mulai setelah '{' awal, jangan mulai menulis di baris bawahnya
%=============================================================================
\abstrakINA{Pada proses penambangan data, data yang digunakan seringkali diberikan kepada pihak lain dan ada kemungkinan privasi di dalam data tersebut tersebar kepada pihak yang tidak berhak. Data privasi tersebut dapat tersebar kepada pihak yang tidak bertanggung jawab dan disalahgunakan. Dalam menghindari hal tersebut, \textit{privacy-preserving data mining} perlu dilakukan. Privasi dapat diartikan sebagai sebuah informasi personal seseorang yang dapat mengidentifikasi sesuatu hal pada orang tersebut. Salah satu cara untuk melakukan \textit{privacy-preserving data mining} adalah mengacak data menggunakan metode \textit{Randomization}. Metode \textit{Randomization} bekerja dengan cara mengacak data tetapi data tersebut masih dapat digunakan untuk penambangan data. Pada penelitian ini dibangun sebuah perangkat lunak yang mengimplementasikan 2 buah teknik yang menggunakan metode \textit{Randomization} yaitu teknik \textit{Random Rotation Perturbation} dan \textit{Random Projection Perturbation}. 

Pengujian dilakukan dengan menerapkan penambangan data klasifikasi dengan algoritma \textit{k-nearest neighbors} dan penambangan data \textit{clustering} dengan algoritma \textit{k-means} masing-masing untuk menhitung akurasi model dan kemiripan hasil \textit{cluster}. Berdasarkan hasil pengujian, model penambangan data yang dilatih dengan \textit{dataset} asli dan \textit{dataset} yang telah diacak dengan teknik \textit{Random Rotation Perturbation} atau \textit{Random Projection Perturbation} memiliki kualitas yang sama atau sangat mirip. Kedua teknik tersebut hanya dapat digunakan untuk data yang bersifat numerik dan khususnya untuk teknik \textit{Random Projection Perturbation} hanya dapat digunakan untuk data yang memenuhi syarat teknik tersebut yaitu jumlah fitur pada data harus cukup banyak.}
\abstrakENG{In the data mining process, the data used is often given to other parties and there is the possibility of privacy in the data being spread to unauthorized parties. The privacy data can be spread to irresponsible and misused parties. To avoid this, privacy-preserving data mining needs to be done. Privacy can be interpreted as a person's personal information that can identify something about that person. One way to do privacy-preserving data mining is to randomize data using the Randomization method. The Randomization method works by randomizing data but the data can still be used for data mining. In this research, a software will be built that implements 2 techniques that use the Randomization method, namely the Random Rotation Perturbation technique and the Random Projection Perturbation.

The test will be carried out by applying classification data mining with the k-nearest neighbors algorithm and clustering data mining with the k-means algorithm respectively to calculate the accuracy of the model and the similarity of the cluster results. Based on the test results, the data mining model that is trained with the original dataset and the dataset that has been randomized with the Random Rotation Perturbation technique or Random Projection Perturbation has the same or very similar quality. Both of these techniques can only be used for numerical data and specifically for the Random Projection Perturbation technique can only be used for data that meet the technical requirement that is the number of features in the data must be quite a lot.} 
%=============================================================================

%_____________________________________________________________________________
%=============================================================================
% 								BAGIAN XI
%=============================================================================
% Kata-kata kunci dan keywords : diletakkan di bawah abstrak (ina dan eng)
% - kunciINA: kata-kata kunci dalam bahasa indonesia
% - kunciENG: keywords in english
% Petunjuk: hilangkan tanda << & >>, dan isi sesuai dengan data anda.
%=============================================================================
\kunciINA{Privasi, \textit{privacy-preserving data mining}, \textit{Randomization}, \textit{Random Rotation Perturbation}, \textit{Random Projection Perturbation}, penambangan data, klasifikasi, \textit{clustering}, \textit{k-nearest neighbors}, \textit{k-means}}
\kunciENG{Privacy, privacy-preserving data mining, Randomization, Random Rotation Perturbation, Random Projection Perturbation, data mining, classification, clustering, k-nearest neighbors, k-means}
%=============================================================================

%_____________________________________________________________________________
%=============================================================================
% 								BAGIAN XII
%=============================================================================
% Persembahan : kepada siapa anda mempersembahkan skripsi ini ...
% Petunjuk: hilangkan tanda << & >>, dan isi sesuai dengan data anda.
%=============================================================================
\untuk{Dipersembahkan untuk Tuhan YME, keluarga tercinta, Ibu Mariskha sebagai dosen pembimbing, teman-teman yang telah membantu dan memberi semangat dalam penyusunan skripsi ini, dan diri sendiri}
%=============================================================================

%_____________________________________________________________________________
%=============================================================================
% 								BAGIAN XIII
%=============================================================================
% Kata Pengantar: tempat anda menuliskan kata pengantar dan ucapan terima 
% kasih kepada yang telah membantu anda bla bla bla ....  
% Petunjuk: hilangkan tanda << & >>, dan isi sesuai dengan data anda.
%=============================================================================
\prakata{Puji syukur penulis panjatkan kepada Tuhan Yang Maha Esa karena atas karunia-Nya, penulis dapat menyelesaikan penyusunan skripsi yang berjudul "Analisis Teknik \textit{Random Rotation Perturbation} dan \textit{Random Projection Perturbation} dalam Mengacak Data untuk Penambangan Data". Selama penyusunan skripsi ini, penulis menghadapi banyak kendala dan berbagai masalah. Penulis menyadari bahwa penyusunan skripsi ini juga tidak terlepas dari bantuan berbagai pihak, baik langsung maupun tidak langsung. Secara khusus, penulis ingin berterima kasih kepada:
\begin{enumerate}
    \item Keluarga yang selalu memberikan dukungan kepada penulis baik berupa doa atau dukungan mental serta materiil.
    \item Ibu Mariskha Tri Adithia, S.Si., M.Sc., PDEng. selaku dosen pembimbing yang telah membimbing penulis dan memberikan dukungan maupun bantuan kepada penulis dalam proses penyusunan skripsi ini.
    \item Ibu Rosa De Lima, M.Kom. dan Ibu Luciana Abednego, M.T. selaku dosen penguji yang telah memberikan kritik dan saran yang membangun sehingga penelitian ini menjadi lebih baik.
    \item Teman-teman penulis di jurusan Teknik Informatika UNPAR angkatan 2016 yang telah menemani penulis dalam menyelesaikan perkuliahan dari awal semester sampai akhir semester penulis selesai menyusun skripsi ini serta teman-teman di luar jurusan yang tidak dapat disebutkan satu persatu.
    \item Teman seperjuangan skripsi, khususnya Kevin Arnold dan Apsari Ayusya Cantika yang berdosen pembimbing sama dengan penulis, bimbingan bersama, saling membantu, dan saling memberikan dukungan satu sama lain selama menyusun skripsi ini.
\end{enumerate}
Penulis menyadari bahwa penelitian ini masih jauh dari kata sempurna. Oleh karena itu, penulis memohon maaf jika terdapat kesalahan. Penulis juga mengharapkan kritik dan saran yang membangun untuk menyempurnakan penelitian ini. Semoga penelitian ini dapat memberi informasi yang bermanfaat dan menjadi inspirasi untuk penelitian-penelitian berikutnya.} 
%=============================================================================

%_____________________________________________________________________________
%=============================================================================
% 								BAGIAN XIV
%=============================================================================
% Tambahkan hyphen (pemenggalan kata) yang anda butuhkan di sini 
%=============================================================================
\hyphenation{ma-te-ma-ti-ka}
\hyphenation{fi-si-ka}
\hyphenation{tek-nik}
\hyphenation{in-for-ma-ti-ka}
%=============================================================================

%_____________________________________________________________________________
%=============================================================================
% 								BAGIAN XV
%=============================================================================
% Tambahkan perintah yang anda buat sendiri di sini 
%=============================================================================
\renewcommand{\vtemplateauthor}{lionov}
\pgfplotsset{compat=newest}
%=============================================================================
