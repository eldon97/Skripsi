\chapter{Kesimpulan dan Saran}
\label{chap:kesimpulan-saran}

Bab ini berisi kesimpulan dari awal hingga akhir penelitian beserta saran untuk penelitian selanjutnya.

\section{Kesimpulan}
\label{sec:kesimpulan}

Kesimpulan yang dapat ditarik dari penelitian ini adalah sebagai berikut.
\begin{itemize}
    \item Teknik \textit{Random Rotation Perturbation} mengacak data dengan cara merotasi setiap titik yang merepresentasikan sebuah objek data pada bidang Euclidean sehingga nilai setiap objek data berubah tetapi jarak Euclidean antara setiap titik dengan titik lainnya tidak berubah. Oleh karena itu, data yang telah diacak tersebut masih dapat digunakan untuk teknik penambangan data yang hanya memanfaatkan jarak Euclidean.
    \item Teknik \textit{Random Projection Perturbation} mengacak data dengan cara memproyeksikan data yang berdimensi cukup besar ke dimensi yang lebih kecil. Teknik ini memanfaatkan \textit{Johnson-Lindenstrauss Lemma} yang menyatakan bahwa titik-titik pada bidang Euclidean \(d\)-dimensi dapat diproyeksikan ke bidang Euclidean yang berdimensi lebih kecil dari \(d\) tetapi jarak Euclidean antara setiap titik tetap terjaga dengan distorsi yang terkontrol tetapi dengan syarat \(d\) harus cukup besar. Oleh karena itu, data yang telah diacak tersebut masih dapat digunakan untuk teknik penambangan data yang hanya memanfaatkan jarak Euclidean.
    \item Perangkat lunak diimplementasikan dengan bahasa pemograman Python. Antarmuka perangkat lunak dibuat dengan bantuan \textit{framework} antarmuka bernama Kivy. Fungsi-fungsi pada perangkat lunak yang berkaitan dengan masukan, keluaran, dan implementasi algoritma dibantu oleh berbagai \textit{library} seperti Numpy, Scipy, Pandas, dan Scikit-learn. Perangkat lunak hanya dapat menerima masukan dataset berupa dokumen \textit{comma-separated values} yang setiap fiturnya bersifat numerik saja.
    \item Teknik \textit{Random Rotation Perturbation} diimplementasikan dengan cara mengolah dataset yang ingin diacak menjadi matriks, lalu melakukan tranformasi translasi pada matriks dataset tersebut dengan cara mengkalikannya dengan matriks translasi acak yang dibuat mengikuti distribusi \textit{uniform} dengan rentang nilai [0, 100]. Kemudian tranformasi rotasi dilakukan pada matriks dataset yang telah ditranslasi dengan cara mengkalikannya dengan matriks rotasi acak yang dibuat mengikuti distribusi Haar dengan bantuan \textit{library} Scipy.
    \item Teknik \textit{Random Projection Perturbation} diimplementasikan dengan cara mengolah dataset yang ingin diacak menjadi matriks, lalu melakukan pemeriksaan persyaratan terlebih dahulu dan menentukan target dimensi yang menjadi dimensi dataset hasil pengacakan. Kemudian apabila persyaratan terpenuhi maka proyeksi dapat dilakukan dengan cara mengkalikan matriks dataset tersebut dengan matriks proyeksi acak yang dibuat mengikuti distribusi normal.
    \item Teknik \textit{Random Rotation Perturbation} terbukti menjaga jarak Euclidean dengan sempurna tanpa ada distorsi dan teknik \textit{Random Projection Perturbation} terbukti menjaga jarak Euclidean dengan distorsi yang sesuai dengan keinginan pengguna.
    \item Berdasarkan pengujian eksperimental yang telah dilakukan, metode \textit{Randomization} mengacak data dan berbagai properti yaitu rata-rata, standar deviasi, nilai terkecil, nilai terbesar, kuartil bawah, kuartil tengah, dan kuartil atas setiap kolom tanpa merusak jarak Euclidean pada data tersebut.
    \item \textit{Dataset} yang diacak dengan teknik \textit{Random Rotation Perturbation} terbukti masih dapat digunakan untuk penambangan data klasifikasi menggunakan teknik \textit{k-nearest neighbors} dengan akurasi model klasifikasi yang sama persis. Sedangkan untuk teknik \textit{Random Projection Perturbation} memiliki akurasi model klasifikasi yang sangat mirip dengan akurasi model yang dilatih dengan \textit{dataset} asli dan jumlah tetangga (nilai variabel \textit{k}) pada model yang memiliki akurasi tertinggi juga berbeda dengan model yang dilatih dengan \textit{dataset} asli.
    \item \textit{Dataset} yang diacak dengan teknik \textit{Random Rotation Perturbation} terbukti masih dapat digunakan untuk penambangan data \textit{clustering} menggunakan teknik \textit{k-means} dengan hasil \textit{cluster} yang sama persis dengan model yang dilatih dengan \textit{dataset} asli. Sedangkan untuk teknik \textit{Random Projection Perturbation} memiliki hasil \textit{cluster} yang sangat mirip dengan model yang dilatih menggunakan \textit{dataset} asli.
    \item Waktu eksekusi dalam proses penambangan data klasifikasi dan \textit{clustering} menggunakan \textit{dataset} yang diacak dengan teknik \textit{Random Rotation Perturbation} tidak memiliki perbedaan yang signifikan dibandingkan penambangan data dengan \textit{dataset} asli. Sedangkan untuk waktu eksekusi dalam proses penambangan data klasifikasi dan \textit{clustering} menggunakan \textit{dataset} yang diacak dengan teknik \textit{Random Projection Perturbation} lebih cepat dibandingkan penambangan data dengan \textit{dataset} asli karena jumlah fitur yang ada pada \textit{dataset} lebih sedikit.
\end{itemize}

\section{Saran}
\label{sec:saran}

Saran untuk penelitian selanjutnya adalah sebagai berikut.
\begin{itemize}
    \item Pada penelitian ini tidak menguji apakah data yang telah diacak berpotensi dapat dikembalikan ke aslinya. Untuk penelitian selanjutnya, metode \textit{Randomization} dapat diuji untuk mengetahui apakah ada potensi data yang telah diacak dengan metode \textit{Randomization} dapat dikembalikan ke aslinya.
    \item Pada penelitian ini metode \textit{Randomization} diuji dengan menggunakan penambangan data biasa, bukan dalam lingkungan \textit{big data}. Untuk penelitian selanjutnya, metode \textit{Randomization} dapat diuji untuk diimplementasikan dalam lingkungan \textit{big data} dan mengukur kecepatan dan kualitas teknik \textit{Random Rotation Perturbation} dan \textit{Random Projection Perturbation}.
    \item Untuk penelitian selanjutnya dapat dilakukan pengujian lebih lanjut lagi terhadap teknik \textit{Random Projection Perturbation} menggunakan data yang lain untuk mengetahui nilai variabel \textit{epsilon} yang tepat sehingga kualitas model penambangan data yang dilatih menggunakan \textit{dataset} yang telah diacak masih dapat ditoleransi perbedaannya dari yang asli dan bagaimana menanggulangi masalah nilai variabel \textit{k} (dimensi minimal) pada teknik \textit{Random Projection Perturbation} semakin menaik seiring model penambangan data dilatih dengan data yang baru.
    \item Untuk penelitian selanjutnya dapat dilakukan eksperimen mengenai perbedaan teknik \textit{Principal Component Analysis} dengan teknik \textit{Random Projection Perturbation} dalam mereduksi dimensi data. Eksperimen dapat bertujuan untuk menganalisa kualitas hasil dari kedua teknik tersebut dan apakah ada potensi sebuah data yang telah direduksi dengan kedua teknik tersebut dapat dikembalikan ke aslinya.
\end{itemize}
