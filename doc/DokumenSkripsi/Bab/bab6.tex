\chapter{Kesimpulan dan Saran}
\label{chap:kesimpulan-saran}

Bab ini berisi kesimpulan dari awal hingga akhir penelitian beserta saran untuk penelitian selanjutnya.

\section{Kesimpulan}
\label{sec:kesimpulan}

Kesimpulan yang dapat ditarik dari penelitian ini adalah sebagai berikut.
\begin{itemize}
    \item Pada penelitian ini, telah dipelajari cara kerja untuk mengimplementasikan dua teknik metode \textit{Randomization} yaitu teknik \textit{Random Rotation Perturbation} dan teknik \textit{Random Projection Perturbation} untuk mengacak data yang akan digunakan untuk penambangan data dengan teknik \textit{k-nearest neighbors} dan \textit{k-means}
    \item Pada penelitian ini, telah diimplementasikan teknik \textit{Random Rotation Perturbation} dan teknik \textit{Random Projection Perturbation} menjadi sebuah perangkat lunak. Perangkat lunak metode \textit{Randomization} ini dapat mengacak data dengan kedua teknik tersebut dengan syarat seluruh nilai pada data harus bersifat numerik dan ada sebuah persyaratan untuk teknik \textit{Random Projection Perturbation} yaitu data yang digunakan harus cukup besar. Perangkat lunak menerima masukan dan menghasilkan keluaran berupa dokumen berjenis \textit{comma-separated values}. Matriks rotasi/proyeksi yang dibuat perangkat lunak dapat disimpan dan digunakan kembali lain kali
    \item Pada penelitian ini, telah dilakukan pengujian fungsional pada perangkat lunak dan hasilnya perangkat lunak berhasil mengimplementasikan metode \textit{Randomization} dengan benar sesuai tujuan masing-masing teknik
    \item Pada penelitian ini, telah dilakukan pengujian eksperimental untuk menguji apakah data yang telah diacak dengan metode \textit{Randomization} masih dapat digunakan untuk penambangan data klasifikasi menggunakan teknik \textit{k-nearest neighbors} dengan akurasi model yang sama atau mirip dan penambangan data \textit{clustering} menggunakan teknik \textit{k-means} dengan model \textit{clustering} yang sama atau mirip dengan menghitung kemiripannya menggunakan metode \textit{Adjusted Rand Index}. Beberapa properti pada data juga diuji untuk melihat apakah ada kesamaan atau bagaimana perbedaan yang terjadi
    \item Berdasarkan pengujian eksperimental yang telah dilakukan, seluruh nilai properti data yaitu rata-rata, standar deviasi, nilai terkecil dan terbesar pada sebuah kolom, kuartil bawah, kuartil tengah, dan kuartil atas sangat berbeda. Hal ini menunjukkan bahwa metode \textit{Randomization} mengacak data dan berbagai properti tersebut tanpa merusak jarak Euclidean pada data tersebut
    \item Berdasarkan pengujian eksperimental yang telah dilakukan, dataset yang diacak dengan teknik \textit{Random Rotation Perturbation} terbukti masih dapat digunakan untuk penambangan data klasifikasi menggunakan teknik \textit{k-nearest neighbors} dengan akurasi model klasifikasi yang sama persis. Waktu eksekusi dalam pembuatan model klasifikasi juga tidak memiliki perbedaan yang signifikan
    \item Berdasarkan pengujian eksperimental yang telah dilakukan, dataset yang diacak dengan teknik \textit{Random Projection Perturbation} terbukti masih dapat digunakan untuk penambangan data klasifikasi menggunakan teknik \textit{k-nearest neighbors} dengan akurasi model klasifikasi yang sangat mirip atau mendekati dengan model yang dilatih dengan dataset asli. Tetapi dengan persyaratan dataset yang ingin diacak harus berdimensi cukup besar sehingga distorsi yang terjadi tidak terlalu signifikan. Akurasi model klasifikasi akan semakin menurun apabila nilai variabel \textit{epsilon} semakin besar karena jarak Euclidean semakin rusak. Waktu eksekusi dalam pembuatan model klasifikasi lebih cepat dibandingkan pembuatan model dengan dataset asli karena jumlah fitur yang digunakan pada dataset lebihs sedikit
    \item Berdasarkan pengujian eksperimental yang telah dilakukan, dataset yang diacak dengan teknik \textit{Random Rotation Perturbation} terbukti masih dapat digunakan untuk penambangan data \textit{clustering} menggunakan teknik \textit{k-means} dengan hasil \textit{cluster} yang sama persis dengan model yang dilatih dengan dataset asli. Ukuran kemiripan dihitung dengan metode \textit{Adjusted Rand Index} dan nilainya adalah sebesar 1.0 yang berarti kedua model tersebut sama persis. Waktu eksekusi dalam pembuatan model \textit{clustering} juga tidak memiliki perbedaan yang signifikan
    \item Berdasarkan pengujian eksperimental yang telah dilakukan, dataset yang diacak dengan teknik \textit{Random Projection Perturbation} terbukti masih dapat digunakan untuk penambangan data \textit{clustering} menggunakan teknik \textit{k-means} dengan hasil \textit{cluster} yang sangat mirip dengan model yang dilatih dengan dataset asli. Ukuran kemiripan dihitung dengan metode \textit{Adjusted Rand Index} dan nilainya sangat mendekati 1.0 yang berarti kedua model tersebut sangat mirip. Tetapi untuk mendapatkan hasil yang baik tersebut ada hal harus diperhatikan yaitu dataset yang digunakan harus cukup besar sehingga distorsi yang terjadi tidak terlalu besar. Waktu eksekusi dalam pembuatan model \textit{clustering} lebih cepat dibandingkan pembuatan model dengan dataset asli karena jumlah fitur yang digunakan pada dataset lebih sedikit
\end{itemize}

\section{Saran}
\label{sec:saran}

Saran untuk penelitian selanjutnya adalah sebagai berikut.
\begin{itemize}
    \item Pada penelitian ini tidak menguji apakah data yang telah diacak berpotensi dapat dikembalikan ke aslinya. Untuk penelitian selanjutnya, metode \textit{Randomization} dapat diuji untuk mengetahui apakah ada potensi data yang telah diacak dengan metode \textit{Randomization} dapat dikembalikan ke aslinya.
    \item Pada penelitian ini metode \textit{Randomization} diuji dengan menggunakan penambangan data biasa, bukan dalam lingkungan \textit{big data}. Untuk penelitian selanjutnya, metode \textit{Randomization} dapat diuji untuk diimplementasikan dalam lingkungan \textit{big data} dan mengukur kecepatan dan kualitas teknik \textit{Random Rotation Perturbation} dan \textit{Random Projection Perturbation}
\end{itemize}
