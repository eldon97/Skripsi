\chapter{Kesimpulan dan Saran}
\label{chap:kesimpulan-saran}

Bab ini berisi kesimpulan dari awal hingga akhir penelitian beserta saran untuk penelitian selanjutnya.

\section{Kesimpulan}
\label{sec:kesimpulan}

Kesimpulan yang dapat ditarik dari penelitian ini adalah sebagai berikut.
\begin{itemize}
    \item Pada penelitian ini telah dipelajari cara kerja untuk mengimplementasikan dua teknik metode \textit{Randomization} yaitu teknik \textit{Random Rotation Perturbation} dan teknik \textit{Random Projection Perturbation} untuk mengacak data yang akan digunakan untuk penambangan data dengan teknik \textit{k-nearest neighbors} dan \textit{k-means}
    \item Pada penelitian ini telah diimplementasikan teknik \textit{Random Rotation Perturbation} dan teknik \textit{Random Projection Perturbation} menjadi sebuah perangkat lunak. Perangkat lunak dapat mengacak data dengan kedua teknik tersebut dengan syarat seluruh nilai pada data harus bersifat numerik dan ada sebuah persyaratan untuk teknik \textit{Random Projection Perturbation} yaitu data yang digunakan harus cukup besar. Perangkat lunak menerima masukan dan menghasilkan keluaran berupa dokumen berjenis \textit{comma-separated values}. Matriks rotasi/proyeksi yang dibuat perangkat lunak dapat disimpan dan digunakan kembali pada lain waktu
    \item Pada penelitian ini telah dilakukan pengujian fungsional pada perangkat lunak dan hasilnya perangkat lunak berhasil mengimplementasikan metode \textit{Randomization} dengan benar sesuai tujuan masing-masing teknik
    \item Berdasarkan pengujian eksperimental yang telah dilakukan, seluruh nilai properti data yaitu rata-rata, standar deviasi, nilai terkecil dan terbesar pada sebuah kolom, kuartil bawah, kuartil tengah, dan kuartil atas sangat berbeda. Hal ini menunjukkan bahwa metode \textit{Randomization} mengacak data dan berbagai properti tersebut tanpa merusak jarak Euclidean pada data tersebut
    \item Dataset yang diacak dengan teknik \textit{Random Rotation Perturbation} terbukti masih dapat digunakan untuk penambangan data klasifikasi menggunakan teknik \textit{k-nearest neighbors} dengan akurasi model klasifikasi yang sama persis. Sedangkan untuk teknik \textit{Random Projection Perturbation} memiliki akurasi model klasifikasi yang sangat mirip atau mendekati dengan model yang dilatih dengan dataset asli.
    \item Dataset yang diacak dengan teknik \textit{Random Rotation Perturbation} terbukti masih dapat digunakan untuk penambangan data \textit{clustering} menggunakan teknik \textit{k-means} dengan hasil \textit{cluster} yang sama persis dengan model yang dilatih dengan dataset asli. Ukuran kemiripan dihitung dengan metode \textit{Adjusted Rand Index} dan nilainya adalah sebesar 1.0 yang berarti kedua model tersebut sama persis. Sedangkan untuk teknik \textit{Random Projection Perturbation} memiliki hasil \textit{cluster} yang sangat mirip dengan model yang dilatih menggunakan dataset asli dan nilai \textit{Adjusted Rand Index}-nya sangat mendekati nilai 1.0 yang berarti kedua model tersebut sangat mirip.
    \item Waktu eksekusi dalam pembuatan model penambangan data menggunakan dataset yang diacak dengan teknik \textit{Random Projection Perturbation} lebih cepat dibandingkan pembuatan model dengan dataset asli karena jumlah fitur yang ada pada dataset lebih sedikit. Sedangkan untuk waktu eksekusi pembuatan model penambangan data menggunakan dataset yang diacak dengan teknik \textit{Random Rotation Perturbation} tidak ada perbedaan yang signifikan dibandingkan pembuatan model dengan dataset asli
    \item Kualitas hasil dari teknik \textit{Random Projection Perturbation} akan menurun dalam 2 kondisi yaitu pertama apabila data direduksi ke dimensi yang lebih kecil lagi (semakin besar dimensinya, semakin kecil distorsi pada jarak Euclidean) dan kedua adalah data yang diacak bertambah banyak. Kondisi kedua dikarenakan oleh nilai variabel \textit{k} (dimensi minimal) berbanding lurus dengan banyaknya data. Oleh karena itu semakin sering model dilatih dengan data baru, ada kemungkinan kualitas model tersebut akan menurun apabila nilai variabel \textit{k} (dimensi minimal) sudah melebihi besar dimensi dataset yang telah diacak
\end{itemize}

\section{Saran}
\label{sec:saran}

Saran untuk penelitian selanjutnya adalah sebagai berikut.
\begin{itemize}
    \item Pada penelitian ini tidak menguji apakah data yang telah diacak berpotensi dapat dikembalikan ke aslinya. Untuk penelitian selanjutnya, metode \textit{Randomization} dapat diuji untuk mengetahui apakah ada potensi data yang telah diacak dengan metode \textit{Randomization} dapat dikembalikan ke aslinya.
    \item Pada penelitian ini metode \textit{Randomization} diuji dengan menggunakan penambangan data biasa, bukan dalam lingkungan \textit{big data}. Untuk penelitian selanjutnya, metode \textit{Randomization} dapat diuji untuk diimplementasikan dalam lingkungan \textit{big data} dan mengukur kecepatan dan kualitas teknik \textit{Random Rotation Perturbation} dan \textit{Random Projection Perturbation}
    \item Untuk penelitian selanjutnya dapat dilakukan pengujian lebih lanjut lagi terhadap teknik \textit{Random Projection Perturbation} menggunakan data yang lain untuk mengetahui nilai variabel \textit{epsilon} yang tepat sehingga kualitas model penambangan data yang dilatih menggunakan dataset yang telah diacak masih dapat ditoleransi perbedaannya dari yang asli dan bagaimana menanggulangi masalah nilai variabel \textit{k} (dimensi minimal) pada teknik \textit{Random Projection Perturbation} semakin menaik seiring model penambangan data dilatih dengan data yang baru
\end{itemize}
