%versi 2 (8-10-2016)
\chapter{Landasan Teori}
\label{chap:teori}

Dalam menjaga privasi data, perlu adanya definisi privasi yang konkrit untuk menentukan data seperti apa yang menjadi privasi. Pada penambangan data, perlu ada teknik yang baik untuk menjaga privasi tidak tersebar kepada orang yang tidak berhak. Ada beberapa teknik untuk menjaga privasi pada penambangan data antara lain modifikasi data dengan metode randomisasi yaitu teknik \textit{Random Rotation Perturbation} dan \textit{Random Projection Perturbation}


\section{Privasi Data}
\label{sec:privasidata} 

Pada umumnya sebuah data bisa dikatakan privasi apabila data tersebut dapat dikaitkan dengan identitas seseorang. Tetapi setiap orang mempunyai kepentingan privasi yang berbeda-beda sehingga definisi dari privasi sulit untuk dijelaskan secara eksak. Oleh karena itu, perlu adanya konsep privasi yang dapat menjadi acuan untuk menentukan data seperti apa yang termasuk privasi atau bukan.

\subsection{Privasi}

Dalam mendefinisikan privasi, sulit untuk mendapatkan definisi yang tepat untuk privasi karena setiap individu mempunyai kepentingan yang berbeda-beda sehingga privasi pada setiap individu dapat berbeda-beda juga. Beberapa definisi privasi telah dikemukakan dan definisi tersebut bermacam-macam berdasarkan konteks, budaya, dan lingkungan.~\cite{stanleyosmar:04:standardppdm} Menurut Warren dan Brandeis pada papernya, mereka mendefinisikan privasi sebagai “\textit{the right to be alone.}”, hak untuk menyendiri. Lalu pada papernya, Westin mendefinisikan privasi sebagai “\textit{the desire of people to choose freely under what circumstances and to what extent they will expose themselves, their attitude, and their behavior to others}”, keinginan orang untuk memilih secara bebas dalam segala situasi dan dalam hal mengemukakan diri mereka, sikap mereka, dan tingkah laku mereka pada orang lain. Schoeman mendefinisikan privasi sebagai “\textit{the right to determine what (personal) information is communicated to others}”, hak untuk menentukan informasi pribadi apa saja yang dikomunikasikan kepada yang lain, atau “\textit{the control an individual has over information about himself or herself.}”, kendali seorang individu terhadap informasi tentang dirinya sendiri. Lalu baru-baru ini, Garfinkel menyatakan bahwa “\textit{privacy is about self-possession, autonomy, and integrity.}”, privasi adalah tentang penguasaan diri sendiri, otonomi, dan integritas. Di samping itu, Rosenberg berpendapat bahwa privasi sebenarnya bukan sebuah hak tetapi sebuah rasa: “\textit{If privacy is in the end a matter of individual taste, then seeking a moral foundation for it -- beyond its role in making social institutions possible that we happen to prize -- will be no more fruitful than seeking a moral foundation for the taste for truffles.}”, intinya setiap orang mempunyai perhatian yang berbeda-beda terhadap privasi mereka sendiri sehingga hal tersebut tergantung apa yang dirasakan oleh setiap individu.

Dari definisi-definisi privasi yang telah disebutkan di atas, dapat disimpulkan bahwa privasi dilihat sebagai konsep sosial dan budaya.~\cite{stanleyosmar:04:standardppdm} Konsep privasi pada suatu lingkungan dapat berbeda dari lingkungan lainnya dan hal ini menyebabkan sulitnya menentukan apakah sebuah data termasuk privasi atau bukan. Oleh karena itu, perlu adanya sebuah standar privasi untuk menentukan data mana yang dapat disebut sebuah privasi. Organisasi National Institute of Standards and Technology dari Amerika Serikat, membuat standar mereka sendiri untuk menentukan informasi seperti apa yang dapat disebut sebagai privasi. Mereka mengemukakan konsep \textit{Personally Identifiable Information} sebagai informasi yang dapat dikatakan personal untuk setiap individu.

\subsection{\textit{Personally Identifiable Information}}

Privasi dapat dikatakan adalah sebuah informasi personal seseorang yang dapat mengidentifikasi suatu hal pada orang tersebut. Konsep yang seringkali dipakai untuk mendeskripsikan informasi personal adalah\textit{Personally Identifiable Information} yang disingkat PII. PII adalah segala informasi mengenai individu yang dikelola oleh sebuah instansi, termasuk segala informasi yang dapat digunakan untuk membedakan atau mengusut identitas seseorang dan juga segala informasi yang berhubungan atau dapat dihubungkan kepada suatu individu, seperti informasi medis, pendidikan, finansial, dan pekerjaan seseorang.~\cite{nist:08:pii} 

Informasi yang termasuk membedakan individu adalah informasi yang dapat mengidentifikasi seorang individu. Beberapa contoh informasi yang mengidentifikasi seorang individu adalah nama, nomor KTP, tempat tanggal lahir, nama ibu kandung, atau catatan medis. Sedangkan, data yang hanya berisi misalkan saldo tabungan tanpa ada informasi lain mengenai identitas seseorang yang berkaitan tidak menyediakan informasi yang cukup untuk mengidentifikasi seorang individu.

Mengusut identitas seseorang adalah proses dari membuat perkiraan tentang aspek spesifik dari aktivitas atau status seseorang. Contohnya adalah sebuah catatan finansial seseorang dapat digunakan untuk memperkirakan aktivitas dari individu tersebut.

Informasi yang berhubungan dapat didefinisikan sebagai informasi yang berkaitan dengan seorang individu yang mana terkait secara logis dengan informasi lain tentang individu tersebut. Contohnya adalah apabila ada dua buah basis data yang memiliki data berbeda dari seorang individu, maka seseorang yang mempunyai akses pada 2 basis data tersebut berpotensi dapat mengaitkan data-data tersebut lalu mengidentifikasi individu yang ada pada data tersebut.

\section{Penambangan Data}
\label{sec:penambangandata}

Pada era teknologi informasi, sangat banyak data terkumpul pada basis data. Data yang masif ini dapat dimanfaatkan untuk menggali informasi penting yang berguna untuk pembuatan keputusan. Proses pada aktivitas ini secara kasar dapat disebut dengan penambangan data.

Penambangan data adalah proses mengekstrak sebuah pola atau sebuah pengetahuan dari kumpulan data yang besar, yang mana dapat direpresentasikan dan diinterpretasikan.~\cite{mendes:17:ppdmieee} Pada penambangan data, teknik \textit{machine learning} dan \textit{pattern recognition} intensif dipakai untuk mendapatkan pola maupun pengetahuan baru dari data. Tujuan utama dari penambangan data adalah untuk membentuk model deskriptif dan prediktif dari suatu data. Model deskriptif berusaha untuk mengubah pola-pola yang ada pada data menjadi deskripsi yang bisa dimengerti oleh orang awam. Sedangkan model prediktif digunakan untuk memprediksi data yang tidak diketahui atau data yang berpotensi muncul di kemudian hari.

Model tersebut biasanya dibuat dengan menggunakan teknik \textit{machine learning}, yang mana terdapat dua teknik \textit{machine learning} yang paling sering dipakai yaitu \textit{classification} dan \textit{clustering}. Subbab berikutnya akan menjelaskan secara singkat kedua teknik tersebut dan contoh penerapannya.

\section{\textit{Privacy Preserving Data Mining}}
\label{sec:ppdm}

Aktivitas penambangan data melibatkan jumlah data yang sangat besar. Data-data yang dipakai memiliki privasi banyak individu di dalamnya. Hal ini berpotensi menyebabkan pelanggaran privasi dalam kasus tidak adanya proteksi yang cukup dan penyalahgunaan privasi data untuk tujuan lain.~\cite{rezaseifi:11:ppdm} Faktor utama pelanggaran privasi pada penambangan data adalah penyalahgunaan data sehingga hal ini dapat merugikan seorang individu maupun sebuah organisasi. Oleh karena itu, ada kebutuhan untuk menghindari penyebaran informasi pribadi yang rahasia maupun pengetahuan lainnya yang dapat diambil dari data yang dipakai untuk aktivitas penambangan data.

Konsep privasi sering kali lebih kompleks dari pada yang dibayangkan. Dalam kasus penambangan data, definisi dari menjaga privasi masih tidak jelas dan hanya sedikit literatur yang mendiskusikan topik ini.


\section{Metode \textit{Randomization}}
\label{sec:metoderandomization}



\subsection{\textit{Random Rotation Perturbation}}



\subsection{\textit{Random Projection Perturbation}}



\section{\LaTeX}
\label{sec:latex}

Mengapa menggunakan \LaTeX{} untuk buku skripsi dan apa keunggulan/kerugiannya bagi mahasiswa dan pembuat template. 

\dtext{13-14}


\section{Template Skripsi FTIS UNPAR}
\label{sec:template}
 
Akan dipaparkan bagaimana menggunakan template ini, termasuk petunjuk singkat membuat referensi, gambar dan tabel.
Juga hal-hal lain yang belum terpikir sampai saat ini. 
 
\dtext{15-16}

\subsection{Tabel}  
Berikut adalah contoh pembuatan tabel. 
Penempatan tabel dan gambar secara umum diatur secara otomatis oleh \LaTeX{}, perhatikan contoh di file bab2.tex untuk melihat bagaimana cara memaksa tabel ditempatkan sesuai keinginan kita.

Perhatikan bawa berbeda dengan penempatan judul gambar gambar, keterangan tabel harus diletakkan di atas tabel!!
Lihat Tabel~\ref{tab:contoh1} berikut ini:

\begin{table}[H] %atau h saja untuk "kira kira di sini"
	\centering 
	\caption{Tabel contoh}
	\label{tab:contoh1}
	\begin{tabular}{cccc}
		\toprule
		& $v_{start}$ & $\mathcal{S}_{1}$ & $v_{end}$\\

		\midrule
		$\tau_{1}$ & 1 & 12& 20\\
		$\tau_{2}$ & 1 &  & 20\\
		$\tau_{3}$ & 1 & 9 & 20\\
		$\tau_{4}$ & 1 &  & 20\\

		\bottomrule
		
	\end{tabular} 
\end{table}
Tabel~\ref{tab:cthwarna1} dan Tabel~\ref{tab:cthwarna2} berikut ini adalah tabel dengan sel yang berwarna dan ada dua tabel yang bersebelahan. 
\begin{table}[H]
	\begin{minipage}[c]{0.49\linewidth}
		\centering
		\caption{Tabel bewarna(1)}
		\label{tab:cthwarna1}
		\begin{tabular}{ccccc}
			\toprule
			 & $v_{start}$ & $\mathcal{S}_{2}$ & $\mathcal{S}_{1}$ & $v_{end}$\\
			
			\midrule
			$\tau_{1}$ & 1 & 5 \cellcolor{green}& 12& 20\\
			$\tau_{2}$ & 1 & 8 \cellcolor{green}& & 20\\
			$\tau_{3}$ & 1 & 2/8/17 \cellcolor{green}& 9 & 20\\
			$\tau_{4}$ & 1 & \cellcolor{red}& & 20\\
			
			\bottomrule

		\end{tabular}
	\end{minipage}
	\begin{minipage}[c]{0.49\linewidth}
		
		\centering 
		\caption{Tabel bewarna(2)}
		\label{tab:cthwarna2}
		\begin{tabular}{ccccc}
			\toprule
			 & $v_{start}$ & $\mathcal{S}_{1}$ & $\mathcal{S}_{2}$ & $v_{end}$\\
			
			\midrule
			$\tau_{1}$ & 1 & 12& 5 \cellcolor{red} &20\\
			$\tau_{2}$ & 1 &  &  8 \cellcolor{green} &20\\
			$\tau_{3}$ & 1 & 9 & 2/8/17 \cellcolor{green} &20\\
			$\tau_{4}$ & 1 &   & \cellcolor{red} &20\\
			
			\bottomrule
		
		\end{tabular}
	\end{minipage}
\end{table}

 
\subsection{Kutipan}
\label{subs:kutipan} 
Berikut contoh kutipan dari berbagai sumber, untuk keterangan lebih lengkap, silahkan membaca file referensi.bib yang disediakan juga di template ini.
Contoh kutipan:
\begin{itemize}
	\item Buku:~\cite{berg:08:compgeom} 
	\item Bab dalam buku:~\cite{kreveld:04:GIS}
	\item Artikel dari Jurnal:~\cite{buchin:13:median}
	\item Artikel dari prosiding seminar/konferensi:~\cite{kreveld:11:median}
	\item Skripsi/Thesis/Disertasi:~\cite{lionov:02:animasi}~\cite{wiratma:10:following}~\cite{wiratma:22:later}
	\item Technical/Scientific Report:~\cite{kreveld:07:watertight}
	\item RFC (Request For Comments):~\cite{RFC1654}
	\item Technical Documentation/Technical Manual:~\cite{Z.500}~\cite{unicode:16:stdv9}~\cite{google:16:and7}
	\item Paten:~\cite{webb:12:comm}
	\item Tidak dipublikasikan:~\cite{wiratma:09:median}~\cite{lionov:11:cpoly}
	\item Laman web:~\cite{erickson:03:cgmodel}  
	\item Lain-lain:~\cite{agung:12:tango}
\end{itemize}    
  
\subsection{Gambar}

Pada hampir semua editor, penempatan gambar di dalam dokumen \LaTeX{} tidak dapat dilakukan melalui proses {\it drag and drop}.
Perhatikan contoh pada file bab2.tex untuk melihat bagaimana cara menempatkan gambar.
Beberapa hal yang harus diperhatikan pada saat menempatkan gambar:
\begin{itemize}
	\item Setiap gambar {\bf harus} diacu di dalam teks (gunakan {\it field} {\sc label})
	\item {\it Field} {\sc caption} digunakan untuk teks pengantar pada gambar. Terdapat dua bagian yaitu yang ada di antara tanda $[$ dan $]$ dan yang ada di antara tanda $\{$ dan $\}$. Yang pertama akan muncul di Daftar Gambar, sedangkan yang kedua akan muncul di teks pengantar gambar. Untuk skripsi ini, samakan isi keduanya.
	\item Jenis file yang dapat digunakan sebagai gambar cukup banyak, tetapi yang paling populer adalah tipe {\sc png} (lihat Gambar~\ref{fig:ularpng}), tipe {\sc jpg} (Gambar~\ref{fig:ularjpg}) dan tipe {\sc pdf} (Gambar~\ref{fig:ularpdf})
	\item Besarnya gambar dapat diatur dengan {\it field} {\sc scale}.
	\item Penempatan gambar diatur menggunakan {\it placement specifier} (di antara tanda  $[$ dan $]$ setelah deklarasi gambar.
	Yang umum digunakan adalah {\bf H} untuk menempatkan gambar {\bf sesuai} penempatannya di file .tex atau  {\bf h} yang berarti "kira-kira" di sini. \\
	Jika tidak menggunakan {\it placement specifier}, \LaTeX{} akan menempatkan gambar secara otomatis untuk menghindari bagian kosong pada dokumen anda.
	Walaupun cara ini sangat mudah, hindarkan terjadinya penempatan dua gambar secara berurutan. 	
	\begin{itemize}
		\item Gambar~\ref{fig:ularpng} ditempatkan di bagian atas halaman, walaupun penempatannya dilakukan setelah penulisan 3 paragraf setelah penjelasan ini.
		\item Gambar~\ref{fig:ularjpg} dengan skala 0.5 ditempatkan di antara dua buah paragraf. Perhatikan penulisannya di dalam file bab2.tex!
		\item Gambar~\ref{fig:ularpdf} ditempatkan menggunakan {\it specifier} {\bf h}.
	\end{itemize}
\end{itemize}
 
\dtext{17-18}
\begin{figure} 
	\centering  
	\includegraphics[scale=1]{ular-png}  
	\caption[Gambar {\it Serpentes} dalam format png]{Gambar {\it Serpentes} dalam format png} 
	\label{fig:ularpng} 
\end{figure} 

\dtext{19-20}
\begin{figure}[H]
	\centering  
	\includegraphics[scale=0.5]{ular-jpg}  
	\caption[Ular kecil]{Ular kecil} 
	\label{fig:ularjpg} 
\end{figure} 
\dtext{21-22}

\begin{figure}[ht] 
	\centering  
	\includegraphics[scale=1]{ular-pdf}  
	\caption[ {\it Serpentes} betina]{ {\it Serpentes} jantan} 
	\label{fig:ularpdf} 
\end{figure} 
 
